The Dutiya-Parakkamabāhu Cullavagga (DP-CV) is a 13th century Pali Buddhist manuscript curated by the National Museum of Colombo, and owned by the Sri Lankan Department of Archaeology. It is the oldest, and arguably most important, Pali manuscript in Sri Lanka.\markdownRendererInterblockSeparator
{}Bhante Sujato of SuttaCentral has proposed a project to digitize this text, described in the document “Dutiya-Parakkamabāhu Cullavagga Transcription Project”. The primary aims of the project are:\markdownRendererInterblockSeparator
{}\markdownRendererUlBegin
\markdownRendererUlItem Review the manuscript to ascertain the state of preservation.\markdownRendererUlItemEnd 
\markdownRendererUlItem Scan the manuscript into high-resolution images.\markdownRendererUlItemEnd 
\markdownRendererUlItem Engage an epigraphic expert to assess the script.\markdownRendererUlItemEnd 
\markdownRendererUlItem Carbon date the manuscript.\markdownRendererUlItemEnd 
\markdownRendererUlItem Have the manuscript carefully typed and proofread.\markdownRendererUlItemEnd 
\markdownRendererUlItem Publish digitally and in print.\markdownRendererUlItemEnd 
\markdownRendererUlItem Document the project in academic journals and conferences.\markdownRendererUlItemEnd 
\markdownRendererUlItem Publicize the project in popular awareness.\markdownRendererUlItemEnd 
\markdownRendererUlEnd \markdownRendererInterblockSeparator
{}The project is a partnership between the following:\markdownRendererInterblockSeparator
{}\markdownRendererUlBegin
\markdownRendererUlItem National Museum of Colombo\markdownRendererUlItemEnd 
\markdownRendererUlItem Sri Lankan Department of Archaeology\markdownRendererUlItemEnd 
\markdownRendererUlItem University of Sri Jayawardenepura\markdownRendererUlItemEnd 
\markdownRendererUlItem SuttaCentral\markdownRendererUlItemEnd 
\markdownRendererUlEnd \markdownRendererInterblockSeparator
{}The partners undertake to work together to successfully complete the digitization of the DP-CV, so that this unique text can be available to scholars and students of Buddhism throughout the world.\relax